\documentclass[oneside]{amsart}
\usepackage{amssymb,amsfonts}
\usepackage{amsthm}

\linespread{1.5}

%%%%%%%%%%%%%%%%%%%%%%%%%%%%%%%%%%%%%%%%%%%%%%%%%%%%%%%%%%%%%%%%%%%%

\begin{document}
\begin{center}
Texas Tech University.  Applied Mathematics Seminar.

\end{center}

\begin{center}

{\LARGE \uppercase{\textbf{
Forchheimer equations in\\ porous media - Part I
}}}

Luan Thach Hoang, Texas Tech University

Wednesday, March 25, 2009

Room: MA 111, Time: 4:00pm

\end{center}

ABSTRACT.
Darcy's law describes the linear relation between
the velocity and gradient of pressure of the fluid. We consider a general non-linear relationship between them. This covers many
models suggested by Forchheimer and investigated by
engineers. For slightly compressible fluids, using this
relation, we derive for the pressure a degenerate non-linear parabolic
equation with the implicit coefficient function. Either non-autonomous
Dirichlet or the time-dependent total flux boundary condition is imposed
on a portion of the boundary. We establish the monotonicity which
guarantees the uniqueness of classical solutions.  Other a prior
estimates, well-posedness, stability and long time dynamics will also be
studied.

This is a joint work with Eugenio Aulisa, Lidia Bloshanskaya and Akif Ibragimov.
\end{document}
