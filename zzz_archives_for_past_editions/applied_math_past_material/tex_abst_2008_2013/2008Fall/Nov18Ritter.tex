\documentclass[oneside]{amsart}
\usepackage{amssymb,amsfonts}
\usepackage{amsthm}

\linespread{2}

%%%%%%%%%%%%%%%%%%%%%%%%%%%%%%%%%%%%%%%%%%%%%%%%%%%%%%%%%%%%%%%%%%%%

\begin{document}
\begin{center}
Texas Tech University. Joint Applied Mathematics and Bio-mathematics Seminar.

\end{center}

\begin{center}

{\LARGE \uppercase{\textbf
{Atherogenesis viewed as an inflammatory instability}}}

Laura R. Ritter, Southern Polytechnic State University

November 18, 2008

Room: MA 108. Time: 4:00pm

\end{center}

ABSTRACT.
Motivated by the disease paradigm articulated by Russell Ross, atherogenesis is viewed as an inflammatory spiral with positive feedback loop involving key cellular and chemical species interacting and reacting within the intimal layer of muscular arteries. The inflammation is modeled through a system of nonlinear reaction-diffusion-convection partial differential equations. The inflammatory spiral is initiated as an instability from a healthy state which is defined to be an equilibrium state devoid of certain key inflammatory markers. Disease initiation is studied through a linear, asymptotic stability analysis of a healthy equilibrium state. Various stability criteria on the parameters of the system as well as the boundary conditions will be given. Included are results accounting for the role of apoptosis on inflammation, cellular and cholesterol transport from the blood stream into the arterial wall, and the possible mitigating effects of anti-oxidants upon transition into inflammatory spiral.

\end{document}
