% abstract for IU talk Oct 17, 2007
\documentclass[twoside]{amsart}
\usepackage{amssymb,amsfonts}
\usepackage{amsthm}
%\usepackage[pagebackref=true]{hyperref}
\usepackage{hyperref}
%----------------------notation-----------------------------

%%%%%%%%%%%%%%%%%%%%%%%%%%%%%%%%%%%%%%%%%%%%%%%%%%%%%%%%%%%%%%%%%%%%
\title{Some problems in geophysical fluid dynamics}
\author{Luan Thach Hoang}
\address{School of Mathematics, University of Minnesota, 127 Vincent Hall, 206 Church St. S.E.,
Minneapolis, MN 55455, U.S.A.}
\email{lthoang@math.umn.edu}

%\date{\today}
%\setcounter{equation}{0}

%%%%%%%%%%%%%%%%%%%%%%%%%%%%%%%%%%%%%%%%%%%%%%%%%%%%%%%%%%%%%%%%%%%%
\begin{document}
%%%%%%%%%%%%%%%%%%%%%%%%%%%%%%%%%%%%%%%%%%%%%%%%%%%%%%%%%%%%%%%%%%%%

%%%%%%%%%%%%%%%%%%%%%%%%%%%%%%%%%%%%%%%%%%%%%%%%%%%%%%%%%%%%%%%%%%%%

\maketitle

%\begin{abstract}
ABSTRACT.
We study mathematical problems in geophysical fluid dynamics.
The Navier-Stokes equations in thin domains are used to model the motion of fluids in the atmosphere and oceans.
The domain considered here has depth of order $\varepsilon$ as $\varepsilon\to 0$. 
The velocity field is subject to the Navier boundary conditions on the non-flat top and bottom boundaries. Roughly speaking, we show that for appropriate forces, if the $H^1$ norm of initial velocity is $O(\varepsilon^{-\frac{1}{2}})$ as $\varepsilon\to 0$, then there exists a unique strong solution for all time.
We also discuss similar results for Boussinesq equations of the oceans which contain the additional unknown temperature and salinity concentration.
Our proofs rely on the study of the dependence of the Stokes operator on $\varepsilon$, 
and the non-linear estimate which has to deal with the non-trivial contributions of the boundary integrals connected with the boundary conditions.


This is a joint work with George Sell.

\end{document}
