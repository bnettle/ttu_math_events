% ----------------------------------------------------------------
% AMS-LaTeX Paper ************************************************
% **** -----------------------------------------------------------
\documentclass{amsart}
\usepackage{graphicx}
% ----------------------------------------------------------------
\vfuzz2pt % Don't report over-full v-boxes if over-edge is small
\hfuzz2pt % Don't report over-full h-boxes if over-edge is small
% THEOREMS -------------------------------------------------------
\newtheorem{thm}{Theorem}[section]
\newtheorem{cor}[thm]{Corollary}
\newtheorem{lem}[thm]{Lemma}
\newtheorem{prop}[thm]{Proposition}
\theoremstyle{definition}
\newtheorem{defn}[thm]{Definition}
\theoremstyle{remark}
\newtheorem{rem}[thm]{Remark}
\numberwithin{equation}{section}
% MATH -----------------------------------------------------------
\newcommand{\norm}[1]{\left\Vert#1\right\Vert}
\newcommand{\abs}[1]{\left\vert#1\right\vert}
\newcommand{\set}[1]{\left\{#1\right\}}
\newcommand{\Real}{\mathbb R}
\newcommand{\eps}{\varepsilon}
\newcommand{\To}{\longrightarrow}
\newcommand{\BX}{\mathbf{B}(X)}
\newcommand{\A}{\mathcal{A}}
% ----------------------------------------------------------------
\begin{document}
 \noindent Title: {\bf Random Walks on Groups,  Switching Systems and Limiting
 Distributions}\medskip

\noindent Speaker: Clyde Martin\medskip

\noindent Date: November 17, 2010
\bigskip

\centerline{ ABSTRACT}

We have extensively studied the systems
 $$\dot{x}=(\delta(t)A+(1-\delta(t))B)x(t),\ \
 \delta(t)\in\{0,1\}$$,
 \begin{eqnarray*}dx&=&(z(t)A+(1-z(t))B)dt\\dz&=&(1-2z)dN_\lambda\end{eqnarray*}
 and $$x_{n+1}=(\delta_1A_1+\cdots+\delta_kA_k)x_n$$ where
 $\delta_i\in\{0,1\}$, $\delta_1+\cdots+\delta_k=1$ and $P(\delta_i=1)=p_i$ and we have
 a fairly good understanding of when these systems are stable. Our
 interest has now moved to an attempt to understand what happens
 when they are not stable. Interesting phenomena occurs when we
 consider a representation of a finite group $G$, i.e. a homomorphism, $L$, from $G$ into
 $Hom(V,V)$.We now consider the system $$X_{n+1}=(\delta_1L(g_1)+\cdots+\delta_kL(g_k))xX_n$$ where
 $\delta_i\in\{0,1\}$, $\delta_1+\cdots+\delta_k=1$,
 $P(\delta_i=1)=p_i$ and $X_0=I$.  Because $X_n=L(g)$ for some
 $g\in G$ the system evolves on the image of the representation.
 Using linear system theory we can calculate all of the moments
 of the limiting distribution and using the fact that the system
 creates a Markov process we can calculate the transition
 probability matrix. A nice interplay between systems theory and
 the work of Perci Diaconis on the role of groups in probability
 and statistics develops. The results  have applications in such
 diverse areas as magic and genetics.








 \end{document}
% ----------------------------------------------------------------
