\documentclass[oneside]{article}
\usepackage{amssymb,amsfonts}
\usepackage{amsthm}

\usepackage{hyperref}

\usepackage{fancyhdr}

\usepackage[none]{hyphenat} 
\usepackage[left=.75 in,top=.75 in,right=.75 in,bottom=.75 in,nohead]{geometry}

\pagestyle{fancy}
\lfoot{}
\cfoot{\url{http://www.math.ttu.edu/~dacao/AnalysisSeminar}}
%\cfoot{\textit{http:/$\!$/www.math.ttu.edu/$\!$\raise-1ex\hbox{{\Large\texttt{\char`\~}}}lhoang/AppliedMath/}}
\rfoot{}
\fancyhead{}
\renewcommand{\headrulewidth}{0pt}
\linespread{1.5}

%%%%%%%%%%%%%%%%%%%%%%%%%%%%%%%%%%%%%%%%%%%%%%%%%%%%%%%%%%%%%%%%%%%%

\newcommand{\talktitle}{Asymptotic expansions in time for solutions of Navier-Stokes equations
of rotating  fluids.}

\newcommand{\talkspeaker}{ \textbf{\sc Luan Hoang }\\ \textit{Texas Tech University}}

\newcommand{\talkdate}{\textbf{Monday, November 26, 2018}}

\newcommand{\timelocation}{\textbf{Room: MATH 109.  Time: 4:00pm.}}

\newcommand{\talkabstract}{
%
}

\begin{document}

%\vspace*{-2cm}
\begin{center}
{\LARGE
Texas Tech University.  Analysis Seminars.
}
\medskip

\textbf{\Huge {\uppercase{\talktitle}} }

{\LARGE
\talkspeaker\\
\talkdate\\
\timelocation
}
\end{center}

\vspace*{10pt}


%\addtolength{\linewidth}{-1cm}
\begin{center}

\fbox{\parbox{\linewidth}{
\begin{center}
\begin{minipage}[c]{.96\linewidth}
{
\vspace*{.25cm}
{\LARGE \textbf{ABSTRACT.}}
{\Large 
We consider the Navier-Stokes equations of viscous, incompressible, rotating fluids in the three-dimensional periodic domains.
In the case the velocity field has zero average, we prove that any Leray-Hopf weak solution admits an asymptotic expansion in Gevrey spaces, as time tends to infinity,
in terms of the the exponentially decaying functions, with the coefficients being ``sinusoidal polynomials'', i.e., combinations of the power, sine and cosine functions of time.
In the general case, by using a transformation of the Galilean type, we show that any global, classical solution admits a similar asymptotic expansion with the coefficients being ``double sinusoidal polynomials''. To deal with the Coriolis force, we use the Poincar\'e wave to transform the equations to a more convenient form, for which the asymptotic expansions can be obtained by using Foias-Saut original ideas and its modification by Martinez and myself.
This is a joint work with Ciprian Foias and Edriss Titi. 
}
\vspace*{.25cm}
}
\end{minipage}
\end{center}
}}

\end{center}
\end{document}
