\documentclass[a4paper, 12pt]{article}
\usepackage{amssymb,amsmath}
\title{\textbf{Boundary Estimates for Solutions to the Two-Phase Parabolic Obstacle Problem}}
\author{Darya E.\,Apushkinskaya\\ 
Saarland University, Saarbr{\"u}cken, Germany}

\begin{document}
\maketitle

Consider the two-phase parabolic obstacle problem with non-trivial Dirichlet condition
\begin{align*}
\Delta u-\partial_tu&=\lambda^+\chi_{\{u>0\}}-\lambda^-\chi_{\{u<0\}} \quad
\text{in}\ Q=\Omega \times (0;T),\\
 u&=\varphi \quad \text{on} \quad \partial_p Q.
\end{align*}
Here $T<+\infty$, $\Omega \subset \mathbb{R}^n$ is a given domain, $\partial_PQ$ denotes the parabolic boundary of $Q$, and $\lambda^{\pm}$ are non-negative constants satisfying $\lambda^++\lambda^->0$. The problem arises as limiting case in the model of temperature control through the interior.

In this talk we discuss the $L^{\infty}$-estimates for the second-order space derivatives $D^2u$ near the parabolic boundary $\partial_p Q$. Observe that the case of general Dirichlet data cannot be reduced to zero ones due to non-linearity and discontinuity at $u=0$ of the right-hand side of the first equation.

The lecture is based on works in collaboration with Nina Uraltseva.
\end{document}