\documentclass[oneside]{amsart}
\usepackage{amssymb,amsfonts}
\usepackage{amsthm}

\linespread{1.5}

%%%%%%%%%%%%%%%%%%%%%%%%%%%%%%%%%%%%%%%%%%%%%%%%%%%%%%%%%%%%%%%%%%%%

\begin{document}

\thispagestyle{empty}
\begin{center}
Texas Tech University.  Applied Mathematics Seminar.

\end{center}

\begin{center}

{\LARGE \uppercase{\textbf{Mathematical model of well productivity index for generalized Forchheimer flows
}}}

Lidia Bloshanskaya, Texas Tech University

Wednesday, September 16, 2009

Room: MA 016, Time: 4:00pm

\end{center}

ABSTRACT.
Motivated by the reservoir engineering concept of the well Productivity Index (PI) we analyze a time dependent functional,  diffusive capacity,  for the solution of IBVPs for non-linear Forchheimer equation.  Basing on the results in our previous work we present an alternative method of estimation of time-dependent diffusive capacity on the basis of the solution of the corresponding time-independent, steady-state boundary value problems.
Two boundary value problems are considered, case of given well-bore pressure and case of given production rate on the boundary of the well. One of the purposes of this presentation is to emphasize the effect of the nonlinear nature of the flow filtration on the value of the diffusive capacity and to investigate structural stability of   non-linear flows with respect to the coefficients of the $g$- polynomial Forchheimer law and the boundary conditions. Theoretical studies are supported by numerical computations for various non-linearities equations and boundary data.


\end{document}
