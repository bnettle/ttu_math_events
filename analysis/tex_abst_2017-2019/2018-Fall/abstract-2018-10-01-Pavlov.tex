\documentclass[oneside]{article}
\usepackage{amssymb,amsfonts}
\usepackage{amsthm}

\usepackage{hyperref}

\usepackage{fancyhdr}

\usepackage[none]{hyphenat} 
\usepackage[left=.75 in,top=.75 in,right=.75 in,bottom=.75 in,nohead]{geometry}

\pagestyle{fancy}
\lfoot{}
\cfoot{\url{http://www.math.ttu.edu/~dacao/AnalysisSeminar}}
%\cfoot{\textit{http:/$\!$/www.math.ttu.edu/$\!$\raise-1ex\hbox{{\Large\texttt{\char`\~}}}lhoang/AppliedMath/}}
\rfoot{}
\fancyhead{}
\renewcommand{\headrulewidth}{0pt}
\linespread{1.5}

%%%%%%%%%%%%%%%%%%%%%%%%%%%%%%%%%%%%%%%%%%%%%%%%%%%%%%%%%%%%%%%%%%%%

\newcommand{\talktitle}{Noncommutative $L^p$-spaces. Part II.}

\newcommand{\talkspeaker}{ \textbf{\sc Dmitri Pavlov }\\ \textit{Texas Tech University}}

\newcommand{\talkdate}{\textbf{Monday, October 01, 2018}}

\newcommand{\timelocation}{\textbf{Room: MATH 109.  Time: 4:00pm.}}

\newcommand{\talkabstract}{
%
}

\begin{document}

%\vspace*{-2cm}
\begin{center}
{\LARGE
Texas Tech University.  Analysis Seminars.
}
\medskip

\textbf{\Huge {\uppercase{\talktitle}} }

{\LARGE
\talkspeaker\\
\talkdate\\
\timelocation
}
\end{center}

\vspace*{10pt}


%\addtolength{\linewidth}{-1cm}
\begin{center}

\fbox{\parbox{\linewidth}{
\begin{center}
\begin{minipage}[c]{.96\linewidth}
{
\vspace*{.25cm}
{\LARGE \textbf{ABSTRACT.}}
{\Large 
Measurable spaces have a noncommutative analog, namely, von Neumann algebras. More specifically, algebras of bounded measurable functions on a
measurable space coincide with commutative von Neumann algebras, and one can develop
an analog of measure theory for noncommutative von Neumann algebras. Examples of noncommutative von Neumann algebras include the algebra of bounded operators on a Hilbert space and the convolution algebra of a locally compact group.

Amazingly, many results of classical analysis and differential
geometry can be transferred to the noncommutative setting.
In this talk I will talk about one such development, namely,
Haagerup's theory of noncommutative $L^p$-spaces,
which can be defined for arbitrary von Neumann algebras.
In the case of bounded operators this definitions recovers the
classical Schatten classes,
but the definition makes sense for other noncommutative von Neumann
algebras as well.

The talk will be accessible to everybody with a basic knowledge of analysis,
including classical $L^p$-spaces as well as bounded operators on Hilbert spaces.
No knowledge of von Neumann algebras will be assumed or required.

}
\vspace*{.25cm}
}
\end{minipage}
\end{center}
}}

\end{center}
\end{document}
