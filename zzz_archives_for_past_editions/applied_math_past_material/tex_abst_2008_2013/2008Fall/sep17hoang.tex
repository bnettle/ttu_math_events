% Very minor change sent from Eric to Luan, Wed Feb 23 10:24:04 PST 2005

\documentclass[twoside]{amsart}
\usepackage{amssymb,amsfonts}
\usepackage{amsthm}
%\usepackage[pagebackref=true]{hyperref}
%\hypersetup{pagebackref=true}

%\usepackage[color,notref,notcite]{showkeys}
%\definecolor{refkey}{gray}{.90}%
%\usepackage[notref,notcite]{showkeys}

%\usepackage{graphics}

\linespread{1.5}

%%%%%%%%%%%%%%%%%%%%%%%%%%%%%%%%%%%%%%%%%%%%%%%%%%%%%%%%%%%%%%%%%%%%
%\setlength{\topmargin}{0in}
%\setlength{\oddsidemargin}{0.1in}
%\setlength{\evensidemargin}{0.1in}
%\setlength{\textheight}{8.25in}
%\setlength{\textwidth}{6.25in}
%%%%%%%%%%%%%%%%%%%%%%%%%%%%%%%%%%%%%%%%%%%%%%%%%%%%%%%%%%%%%%%%%%%%
%%%%%%%%%%%%%%%%%%%%%%%%%%%%%%%%%%%%%%%%%%%%%%%%%%%%%%%%%%%%%%%%%%%

\begin{document}
\begin{center}
Texas Tech University. Applied Mathematics Seminar.

\end{center}

\begin{center}

{\LARGE \uppercase{\textbf{Navier-Stokes equations in thin domains with Navier friction boundary conditions}}}

Luan Hoang, Texas Tech University

September 17, 2008

Room: MA 112, Time: 4:00pm

\end{center}


%\begin{abstract}
ABSTRACT.
We study some problems in geophysical fluids dynamics.
Our focus is the Navier-Stokes equations in a 3D domain $\Omega_\varepsilon$ with non-trivial topography and the depth of order $0(\varepsilon)$, as $\varepsilon\to 0$.
The velocity field is subject to the Navier friction boundary condition on the bottom and top boundaries of $\Omega_\varepsilon$.
Unlike our previous work, here we consider the case when the friction coefficients are of exact order $\varepsilon^\delta$, for $\delta\in[0,1]$, and no conditions are imposed on the domains.
It is shown that if the initial data, resp., the body force, belongs to a large set of $H^1(\Omega_\varepsilon)$, resp., $L^2(\Omega_\varepsilon)$, then the strong solution of the Navier-Stokes equations exists for all time.
For the proof, we establish a uniform Korn inequality without any restrictions on the domains; study of the dependence of the Stokes operator on $\varepsilon$;
and obtain a strong non-linear estimate in which we analyze the interactions between the boundary condition and the inertial term in the Navier-Stokes equations.
\end{document}

%----------------------notation-----------------------------

