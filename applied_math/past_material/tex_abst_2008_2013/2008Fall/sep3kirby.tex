\documentclass[oneside]{amsart}
\usepackage{amssymb,amsfonts}
\usepackage{amsthm}

\linespread{2}

%%%%%%%%%%%%%%%%%%%%%%%%%%%%%%%%%%%%%%%%%%%%%%%%%%%%%%%%%%%%%%%%%%%%

\begin{document}
\begin{center}
Texas Tech University. Applied Mathematics Seminar.

\end{center}

\begin{center}

{\LARGE \uppercase{\textbf{Transforming finite elements}}}

Rob Kirby, Texas Tech University

September 3, 2008

Room: MA 108, Time: 4:00pm

\end{center}


%\begin{abstract}
ABSTRACT.
Typically, finite element codes construct basis functions on a mesh
by mapping basis functions from a fixed reference element to each element
of the domain.  This works very well for classic Lagrange elements and almost
no others.  In this talk, I will show what goes wrong with "the rest" of the elements,
tie this to classic finite element notions such as interpolation equivalence, and
show how such approximation-theoretic ideas provide structural information on
how to transform "most" elements.
\end{document}
