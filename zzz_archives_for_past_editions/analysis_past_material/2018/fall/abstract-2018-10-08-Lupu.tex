\documentclass[oneside]{article}
\usepackage{amssymb,amsfonts}
\usepackage{amsthm}

\usepackage{hyperref}

\usepackage{fancyhdr}

\usepackage[none]{hyphenat} 
\usepackage[left=.75 in,top=.75 in,right=.75 in,bottom=.75 in,nohead]{geometry}

\pagestyle{fancy}
\lfoot{}
\cfoot{\url{http://www.math.ttu.edu/~dacao/AnalysisSeminar}}
%\cfoot{\textit{http:/$\!$/www.math.ttu.edu/$\!$\raise-1ex\hbox{{\Large\texttt{\char`\~}}}lhoang/AppliedMath/}}
\rfoot{}
\fancyhead{}
\renewcommand{\headrulewidth}{0pt}
\linespread{1.5}

%%%%%%%%%%%%%%%%%%%%%%%%%%%%%%%%%%%%%%%%%%%%%%%%%%%%%%%%%%%%%%%%%%%%

\newcommand{\talktitle}{Analytic aspects of the Riemann zeta and multiple zeta values}

\newcommand{\talkspeaker}{ \textbf{\sc Cezar Lupu }\\ \textit{Texas Tech University}}

\newcommand{\talkdate}{\textbf{Monday, October 08, 2018}}

\newcommand{\timelocation}{\textbf{Room: MATH 109.  Time: 4:00pm.}}

\newcommand{\talkabstract}{
%
}

\begin{document}

%\vspace*{-2cm}
\begin{center}
{\LARGE
Texas Tech University.  Analysis Seminars.
}
\medskip

\textbf{\Huge {\uppercase{\talktitle}} }

{\LARGE
\talkspeaker\\
\talkdate\\
\timelocation
}
\end{center}

%\vspace*{10pt}


%\addtolength{\linewidth}{-1cm}
\begin{center}

\fbox{\parbox{\linewidth}{
\begin{center}
\begin{minipage}[c]{0.96\linewidth}
{
\vspace*{.25cm}
{\LARGE \textbf{ABSTRACT.}}
{\Large In this talk I shall introduce the Riemann zeta and multiple zeta values. Although they look rather simple, it turns out that the Riemann zeta and multiple zeta values play a very important role at the interface of analysis, number theory, geometry and physics with applications ranging from periods of mixed Tate motives to evaluating Feynman integrals in quantum field theory. 

The talk will be divided into two parts. The first part deals with rational zeta series representations of Apery's constant $\zeta(3)$ and the evaluation of some more general rational zeta series involving binomial coefficients in terms of values of Riemann zeta and Dirichlet beta functions. One of the main ingredients used in the derivation of such representations is the Clausen acceleration formula as well as power series of trigonometric functions. In particular cases, we recover some well-known representations of $\pi$. 

%The second part will be devoted to multiple zeta values (Euler-Zagier sums). They were introduced independently by Hoffman and Zagier in 1992 and they play a crucial role at the interface between analysis, number theory, combinatorics, algebra and physics.  The central part of the talk will be played by Zagier's formula for the multiple zeta values, $\zeta(2, 2, \ldots, 2, 3, 2, 2,\ldots, 2)$. Zagier's formula is a remarkable example of both strength and the limits of the motivic formalism used by Brown in proving Hoffman's conjecture where the motivic argument does not give us a precise value for the special multiple zeta values $\zeta(2, 2, \ldots, 2, 3, 2, 2,\ldots, 2)$ as rational linear combinations of products $\zeta(m)\pi^{2n}$ with $m$ odd. 
%
%The formula is proven indirectly by computing the generating functions of both sides in closed form and then showing that both are entire functions of exponential growth and that they agree at sufficiently many points to force their equality.  
%By using the Taylor series of integer powers of arcsin function and a related result about expressing rational zeta series involving $\zeta(2n)$ as a finite sum of $\mathbb{Q}$-linear combinations of odd zeta values and powers of $\pi$, we derive a new and direct proof of Zagier's formula in the special case $\zeta(2, 2, \ldots, 2, 3)$. We discuss similar results for multiple Hurwitz zeta values.
}
\vspace*{.25cm}
}
\end{minipage}
\end{center}
}}

\end{center}
\begin{center}

\fbox{\parbox{\linewidth}{
\begin{center}
\begin{minipage}[c]{0.99\linewidth}
{
%\vspace*{.25cm}
{\Large 
%In this talk I shall introduce the Riemann zeta and multiple zeta values. Although they look rather simple, it turns out that the Riemann zeta and multiple zeta values play a very important role at the interface of analysis, number theory, geometry and physics with applications ranging from periods of mixed Tate motives to evaluating Feynman integrals in quantum field theory. 
%
%The talk will be divided into two parts. The first part deals with rational zeta series representations of Apery's constant $\zeta(3)$ and the evaluation of some more general rational zeta series involving binomial coefficients in terms of values of Riemann zeta and Dirichlet beta functions. One of the main ingredients used in the derivation of such representations is the Clausen acceleration formula as well as power series of trigonometric functions. In particular cases, we recover some well-known representations of $\pi$. 

The second part will be devoted to multiple zeta values (Euler-Zagier sums). They were introduced independently by Hoffman and Zagier in 1992 and they play a crucial role at the interface between analysis, number theory, combinatorics, algebra and physics.  The central part of the talk will be played by Zagier's formula for the multiple zeta values, $\zeta(2, 2, \ldots, 2, 3, 2, 2,\ldots, 2)$. Zagier's formula is a remarkable example of both strength and the limits of the motivic formalism used by Brown in proving Hoffman's conjecture where the motivic argument does not give us a precise value for the special multiple zeta values $\zeta(2, 2, \ldots, 2, 3, 2, 2,\ldots, 2)$ as rational linear combinations of products $\zeta(m)\pi^{2n}$ with $m$ odd. 


The formula is proven indirectly by computing the generating functions of both sides in closed form and then showing that both are entire functions of exponential growth and that they agree at sufficiently many points to force their equality.  
By using the Taylor series of integer powers of arcsin function and a related result about expressing rational zeta series involving $\zeta(2n)$ as a finite sum of $\mathbb{Q}$-linear combinations of odd zeta values and powers of $\pi$, we derive a new and direct proof of Zagier's formula in the special case $\zeta(2, 2, \ldots, 2, 3)$. We discuss similar results for multiple Hurwitz zeta values.
}
\vspace*{.25cm}
}
\end{minipage}
\end{center}
}}

\end{center}
\end{document}
