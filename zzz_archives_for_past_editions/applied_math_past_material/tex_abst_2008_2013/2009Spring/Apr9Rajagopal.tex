\documentclass[oneside]{amsart}
\usepackage{amssymb,amsfonts}
\usepackage{amsthm}

\linespread{1.5}

%%%%%%%%%%%%%%%%%%%%%%%%%%%%%%%%%%%%%%%%%%%%%%%%%%%%%%%%%%%%%%%%%%%%

\begin{document}
\begin{center}
Texas Tech University.  Applied Mathematics Seminar.

COLLOQUIUM TALK
\end{center}

\begin{center}

{\LARGE \uppercase{\textbf{
A hierarchy of models for the flow of fluids through porous solids
}}}

Kumbakonam Rajagopal, Texas A\&M University

Thursday, April 9, 2009

Room: MA 114, Time: 4:00pm

\end{center}

ABSTRACT.
The celebrated equations due to Fick and Darcy are approximations that can be obtained systematically on the basis of numerous assumptions within the context of Mixture Theory; these equations however not having been developed in such a manner by Fick or Darcy.  Relaxing the assumptions made in deriving these equations via mixture theory, selectively, leads to a hierarchy of mathematical models and it can be shown that popular models due to Forchheimer, Brinkman, Biot and many others can be obtained via appropriate approximations to the equations governing the flow of interacting continua.  It is shown that a variety of other generalizations are possible in addition to those that are currently in favor, and these might be appropriate for describing numerous  interesting technological applications, e.g., enhanced oil recovery.

\end{document}
