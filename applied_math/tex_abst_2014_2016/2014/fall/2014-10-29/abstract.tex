\documentclass[oneside]{article}

\usepackage{amssymb,amsfonts}
\usepackage{amsthm}
\usepackage{palatino}
\usepackage{hyperref}
\usepackage{graphicx}
\usepackage{color}
\usepackage{fancyhdr}
\usepackage[none]{hyphenat} 
\usepackage[left=.75 in,top=.75 in,right=.75 in,bottom=.75 in,nohead]{geometry}

\pagestyle{fancy}
\lfoot{}
\cfoot{\url{http://www.math.ttu.edu/~gbornia/AppliedMath/}}
\rfoot{}
\fancyhead{}
\renewcommand{\headrulewidth}{0pt}

\linespread{1.5}


%%%%%%%%%%%%%%%%%%%%%%%%%%%%%%%%%%%%%%%%%%%%%%%%%%%%%%%%%%%%%%%%%%%%

\newcommand{\talktitle}{ Hypergeometric functions \\ in the p-adic setting }

\newcommand{\talkspeaker}{ \textbf{\sc Dermot McCarthy} \\ \textit{Texas Tech University}}

\newcommand{\talkdate}{\textbf{Wednesday, October 29, 2014}}

\newcommand{\timelocation}{\textbf{Room: MATH 114.  Time: 4:00pm.}}

\newcommand{\talkabstract}{
Classical hypergeometric functions appear in many areas of mathematics and possess some interesting properties.
In this talk, we will describe recent work on constructing an analogous function in the p-adic setting. 
We will discuss this function's application to counting points on algebraic varieties and its relationship with Fourier coefficients of modular forms.
}

\begin{document}

%\thispagestyle{empty}

\vspace*{-2cm}
\begin{center}
{\LARGE Texas Tech University - Department of Mathematics and Statistics }

\vspace{0.2cm}
{\LARGE Seminar in Applied Mathematics }
\vspace{0.2cm}


\medskip

\textbf{\Huge {\uppercase{\talktitle}} }

{\LARGE
\talkspeaker\\
\talkdate\\
\timelocation
}

\vspace*{10pt}

% \includegraphics[width=2.5in]{macine-biscotti.jpg}
\end{center}

%\vspace*{10pt}


%\addtolength{\linewidth}{-1cm}
\begin{center}

\fbox{\parbox{\linewidth}{
\begin{center}
\begin{minipage}[c]{.96\linewidth}
{
\vspace*{.25cm}
{\LARGE \textbf{ABSTRACT.}}
{\large \talkabstract}
\vspace*{.25cm}
}
\end{minipage}



\end{center}
}}

\end{center}
\end{document}
