\documentclass[oneside]{amsart}
\usepackage{amssymb,amsfonts}
\usepackage{amsthm}

\linespread{1.5}

%%%%%%%%%%%%%%%%%%%%%%%%%%%%%%%%%%%%%%%%%%%%%%%%%%%%%%%%%%%%%%%%%%%%

\begin{document}
\begin{center}
Texas Tech University.  Applied Mathematics Seminar.

\end{center}

\begin{center}

{\LARGE \uppercase{\textbf{Forchheimer equations in porous media: Part II
}}}

Luan Hoang, Texas Tech University

Wednesday, September 9, 2009

Room: MA 016, Time: 4:00pm

\end{center}

ABSTRACT.
This presentation focuses on the stability of non-linear flows of
slightly compressible fluids in porous media not adequately described by
Darcy's law. We study a class of generalized nonlinear momentum
equations which covers three well-known Forchheimer equations, the
so-called two-term, power, and three-term laws. The generalized
Forchheimer equation is inverted to a non-linear Darcy equation with
implicit permeability tensor depending on the pressure gradient. This
results in a degenerate parabolic equation for the pressure. Two classes
of boundary conditions are considered, given pressure and given total
flux. The uniqueness, Lyapunov and asymptotic stability  of the
solutions, and their continuous dependence on the boundary data are
analyzed. 

This is joint work with Aulisa Eugenio, Lidia Bloshanskya and Akif
Ibragimov.
\end{document}
