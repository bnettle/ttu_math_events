\documentclass[oneside]{article}
\usepackage{amssymb,amsfonts}
\usepackage{amsthm}

\usepackage{hyperref}

\usepackage{fancyhdr}

\usepackage[none]{hyphenat} 
\usepackage[left=.75 in,top=.75 in,right=.75 in,bottom=.75 in,nohead]{geometry}

\pagestyle{fancy}
\lfoot{}
\cfoot{\url{http://www.math.ttu.edu/~dacao/AnalysisSeminar}}
%\cfoot{\textit{http:/$\!$/www.math.ttu.edu/$\!$\raise-1ex\hbox{{\Large\texttt{\char`\~}}}lhoang/AppliedMath/}}
\rfoot{}
\fancyhead{}
\renewcommand{\headrulewidth}{0pt}
\linespread{1.5}

%%%%%%%%%%%%%%%%%%%%%%%%%%%%%%%%%%%%%%%%%%%%%%%%%%%%%%%%%%%%%%%%%%%%

\newcommand{\talktitle}{Exercises on the theme of Continuous Symmetrization}

\newcommand{\talkspeaker}{ \textbf{\sc Alexander Solynin }\\ \textit{Texas Tech University}}

\newcommand{\talkdate}{\textbf{Monday, January 29, 2018}}

\newcommand{\timelocation}{\textbf{Room: MATH 108.  Time: 4:00pm.}}

\newcommand{\talkabstract}{
%
}

\begin{document}

%\vspace*{-2cm}
\begin{center}
{\LARGE
Texas Tech University.  Analysis Seminars.
}
\medskip

\textbf{\Huge {\uppercase{\talktitle}} }

{\LARGE
\talkspeaker\\
\talkdate\\
\timelocation
}
\end{center}

\vspace*{10pt}


%\addtolength{\linewidth}{-1cm}
\begin{center}

\fbox{\parbox{\linewidth}{
\begin{center}
\begin{minipage}[c]{.96\linewidth}
{
\vspace*{.25cm}
{\LARGE \textbf{ABSTRACT.}}
{\Large
 The original  \emph{Symmetrization}
(nowadays known as ``Steiner symmetrization'') was introduced by
Jacob Steiner in his attempt to prove classical isoperimetric
inequality in 1836. It was used by many authors to prove numerous
functional inequalities and to solve numerous extremal problems in
Mathematics and Physics.  It is well known that Steiner
symmetrization changes sets and functions, and therefore their
characteristics, globally. Thus, George Pol\'{y}a and Gabo Szeg\"{o} asked in 1951 whether or not it is possible to introduce a \textbf{continuous version} of Steiner symmetrization which has
the same effect as Steiner's symmetrization along the whole path
of transformation? Pol\'{y}a and Szeg\"{o} themselves suggested
such a transformation for the case of convex domains.

In this talk I will discuss this
Pol\'{y}a and Szeg\"{o} continuous transformation and also show
how their continuous symmetrization can be applied to solve
particular problems in Complex Analysis and PDE's.
}
\vspace*{.25cm}
}
\end{minipage}
\end{center}
}}

\end{center}
\end{document}
