\documentclass[oneside]{article}
\usepackage{amssymb,amsfonts}
\usepackage{amsthm}

\usepackage{hyperref}

\usepackage{fancyhdr}

\usepackage[none]{hyphenat} 
\usepackage[left=.75 in,top=.75 in,right=.75 in,bottom=.75 in,nohead]{geometry}

\pagestyle{fancy}
\lfoot{}
\cfoot{\url{http://www.math.ttu.edu/~lhoang/AnalysisSeminar/}}
%\cfoot{\textit{http:/$\!$/www.math.ttu.edu/$\!$\raise-1ex\hbox{{\Large\texttt{\char`\~}}}lhoang/AppliedMath/}}
\rfoot{}
\fancyhead{}
\renewcommand{\headrulewidth}{0pt}
\linespread{1.5}

%%%%%%%%%%%%%%%%%%%%%%%%%%%%%%%%%%%%%%%%%%%%%%%%%%%%%%%%%%%%%%%%%%%%

\newcommand{\talktitle}{Extremal partitioning of the plane and space. Part II. }

\newcommand{\talkspeaker}{ \textbf{\sc Alexander Solynin }\\ \textit{Texas Tech University}}

\newcommand{\talkdate}{\textbf{Monday, February 6, 2017}}

\newcommand{\timelocation}{\textbf{Room: MATH 010.  Time: 4:00pm.}}

\newcommand{\talkabstract}{
%
}

\begin{document}

%\vspace*{-2cm}
\begin{center}
{\LARGE
Texas Tech University.  Analysis Seminars.
}
\medskip

\textbf{\Huge {\uppercase{\talktitle}} }

{\LARGE
\talkspeaker\\
\talkdate\\
\timelocation
}
\end{center}

\vspace*{10pt}


%\addtolength{\linewidth}{-1cm}
\begin{center}

\fbox{\parbox{\linewidth}{
\begin{center}
\begin{minipage}[c]{.96\linewidth}
{
\vspace*{.25cm}
{\LARGE \textbf{ABSTRACT.}}
{\Large
I will discuss the problem of partitioning of the plane and three dimensional space into a system of non-overlapping condensers. Main question here is how to find systems of condensers which minimize certain energy functionals. Equivalently, this problem can be stated as follows: Identify all systems $\{D_1,\ldots, D_n\}$ of non-overlapping  ring domains which provide the maximal value for the  weighted sum of their moduli $\sum \alpha_k^2 {\rm mod}(D_k)$ under certain topological assumptions. }

\vspace*{10pt}
{\Large
In the two-dimensional case the solution of this problem is known for a long time and is given in terms of quadratic differentials. As concerns three-dimensional space, very little is known about geometry of possible extremal configurations and analytic properties of the weighted sums of moduli.
}
\vspace*{.25cm}
}
\end{minipage}
\end{center}
}}

\end{center}
\end{document}
