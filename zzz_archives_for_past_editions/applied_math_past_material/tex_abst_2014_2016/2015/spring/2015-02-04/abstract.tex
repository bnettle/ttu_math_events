\documentclass[oneside]{article}

\usepackage{amssymb,amsfonts}
\usepackage{amsthm}
\usepackage{palatino}
\usepackage[hidelinks]{hyperref}
\usepackage{graphicx}
\usepackage{color}
\usepackage{fancyhdr}
\usepackage[none]{hyphenat} 
\usepackage[left=.75 in,top=.75 in,right=.75 in,bottom=.75 in,nohead]{geometry}

\pagestyle{fancy}
\lfoot{}
\cfoot{\url{http://www.math.ttu.edu/~gbornia/AppliedMath/}}
\rfoot{}
\fancyhead{}
\renewcommand{\headrulewidth}{0pt}

\linespread{1.5}


%%%%%%%%%%%%%%%%%%%%%%%%%%%%%%%%%%%%%%%%%%%%%%%%%%%%%%%%%%%%%%%%%%%%

\newcommand{\talktitle}{General Forchheimer-Ward equations for compressible fluids}

\newcommand{\talkspeaker}{ \textbf{\sc Emine Celik} \\ \textit{Texas Tech University}}

\newcommand{\talkdate}{\textbf{Wednesday, February 4, 2015}}

\newcommand{\timelocation}{\textbf{Room: MATH 111.  Time: 4:00pm.}}

\newcommand{\talkabstract}{
We study generalized Forchheimer flows in porous media for compressible fluids including isentropic gases and slightly compressible fluids.
By using J.C. Ward's dimension analysis we derive a doubly nonlinear parabolic equation for appropriately defined ``pseudo-pressure".
The volumetric flux boundary condition is then converted naturally to a time-dependent Robin-type boundary condition.
We establish both interior and global $L^\infty$-estimates for the pseudo-pressure in terms of the initial and boundary data.
The proofs rely upon a modification of the Moser's iteration and a version of trace theorem suitable for the considered Robin-type boundary condition.
This is joint work with Luan Hoang and Thinh Kieu.
}

\begin{document}

%\thispagestyle{empty}

\vspace*{-2cm}
\begin{center}
{\LARGE Texas Tech University - Department of Mathematics and Statistics }

\vspace{0.2cm}
{\LARGE Seminar in Applied Mathematics }
\vspace{0.2cm}


\medskip

\textbf{\Huge {\uppercase{\talktitle}} }

{\LARGE
\talkspeaker\\
\talkdate\\
\timelocation
}

\vspace*{10pt}

% \includegraphics[width=2.5in]{macine-biscotti.jpg}
\end{center}

%\vspace*{10pt}


%\addtolength{\linewidth}{-1cm}
\begin{center}

\fbox{\parbox{\linewidth}{
\begin{center}
\begin{minipage}[c]{.96\linewidth}
{
\vspace*{.25cm}
{\LARGE \textbf{ABSTRACT.}}
{\large \talkabstract}
\vspace*{.25cm}
}
\end{minipage}



\end{center}
}}

\end{center}
\end{document}
