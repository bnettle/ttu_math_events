\documentclass[oneside]{amsart}
\usepackage{amssymb,amsfonts}
\usepackage{amsthm}

\linespread{1.5}

%%%%%%%%%%%%%%%%%%%%%%%%%%%%%%%%%%%%%%%%%%%%%%%%%%%%%%%%%%%%%%%%%%%%

\begin{document}
\begin{center}
Texas Tech University. Applied Mathematics Seminar.

\end{center}

\begin{center}

{\LARGE \uppercase{\textbf{A discrete-time West Nile virus epidemic model }}}

Sophia Jang, Texas Tech University

October 8, 2008

Room: MA 112, Time: 4:00pm

\end{center}


%\begin{abstract}
ABSTRACT.We present a discrete-time West Nile
virus model, consisting of two interacting populations, the vector
and the avian populations. The avian population is classified into
susceptible, infective, and recovered classes while an individual
vector is either susceptible or infective. The transmission of the
disease is assumed to be only by mosquitoes bites and vertical
transmission in the vector population. The model behavior depends
on a lumped parameter $R_0$. The disease will die out if $R_0<1$
and can persist if $R_0>1$.
\end{document}
