% Very minor change sent from Eric to Luan, Wed Feb 23 10:24:04 PST 2005

\documentclass[twoside]{amsart}
\usepackage{amssymb,amsfonts}
\usepackage{amsthm}
%\usepackage[pagebackref=true]{hyperref}
%\hypersetup{pagebackref=true}

%\usepackage[color,notref,notcite]{showkeys}
%\definecolor{refkey}{gray}{.90}%
%\usepackage[notref,notcite]{showkeys}

%\usepackage{graphics}

\linespread{1.3}

%%%%%%%%%%%%%%%%%%%%%%%%%%%%%%%%%%%%%%%%%%%%%%%%%%%%%%%%%%%%%%%%%%%%
%\setlength{\topmargin}{0in}
%\setlength{\oddsidemargin}{0.1in}
%\setlength{\evensidemargin}{0.1in}
%\setlength{\textheight}{8.25in}
%\setlength{\textwidth}{6.25in}
%%%%%%%%%%%%%%%%%%%%%%%%%%%%%%%%%%%%%%%%%%%%%%%%%%%%%%%%%%%%%%%%%%%%
%%%%%%%%%%%%%%%%%%%%%%%%%%%%%%%%%%%%%%%%%%%%%%%%%%%%%%%%%%%%%%%%%%%

%----------------------notation-----------------------------

%%%%%%%%%%%%%%%%%%%%%%%%%%%%%%%%%%%%%%%%%%%%%%%%%%%%%%%%%%%%%%%%%%%%
\title{Incompressible fluids in thin domains with Navier friction boundary conditions}
\author{Luan Thach Hoang}
\address{School of Mathematics, University of Minnesota, 127 Vincent Hall, 206 Church St. S.E.,
Minneapolis, MN 55455, U.S.A.}
\email{lthoang@math.umn.edu}
%\address{$^3$ Department of Mathematics, Arizona State University, Tempe, AZ 85287-1804, U.S.A.}
%\email{byn@stokes.la.asu.edu}

%\date{\today}
%\setcounter{equation}{0}

%%%%%%%%%%%%%%%%%%%%%%%%%%%%%%%%%%%%%%%%%%%%%%%%%%%%%%%%%%%%%%%%%%%%
\begin{document}
%%%%%%%%%%%%%%%%%%%%%%%%%%%%%%%%%%%%%%%%%%%%%%%%%%%%%%%%%%%%%%%%%%%%

%%%%%%%%%%%%%%%%%%%%%%%%%%%%%%%%%%%%%%%%%%%%%%%%%%%%%%%%%%%%%%%%%%%%

\maketitle

%\begin{abstract}
ABSTRACT. 
This study is motivated by problems arising in oceanic dynamics.
Our focus is the Navier-Stokes equations in a 3D domain $\Omega_\varepsilon$, whose thickness is of order $0(\varepsilon)$
as $\varepsilon\to 0$, having non-trivial topography.
The velocity field is subject to the Navier friction boundary condition on the bottom and top boundaries of $\Omega_\varepsilon$
with the friction coefficients $\gamma_0$ and $\gamma_1$, respectively.
Assume that $\gamma_0$ and $\gamma_1$ are of exact order $\varepsilon^\delta$, for $\delta\in[0,1]$.
It is shown that if the initial data, resp., the body force, belongs to a large set of $H^1(\Omega_\varepsilon)$, resp., $L^2(\Omega_\varepsilon)$, then the strong solution of the Navier-Stokes equations exists for all time.
Our proofs rely on the study of the dependence of the Stokes operator on $\varepsilon$, 
and the non-linear estimate which has to deal with the interactions between the boundary condition and the inertial term in the Navier-Stokes equations.
%\begin{thebibliography}{100}

%\bibitem{FS87}
%    C. Foias, J. C. Saut, \emph{Linearization and normal form of the
%    Navier--Stokes equations with potential forces}, Ann. Inst
%    H. Poincar\'e, Anal. Non Lin\'eaire \textbf{4} (1987), 1--47.

%\end{thebibliography}
\end{document}
