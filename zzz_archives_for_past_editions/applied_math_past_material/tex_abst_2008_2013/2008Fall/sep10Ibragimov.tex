\documentclass[oneside]{amsart}
\usepackage{amssymb,amsfonts}
\usepackage{amsthm}

\linespread{2}

%%%%%%%%%%%%%%%%%%%%%%%%%%%%%%%%%%%%%%%%%%%%%%%%%%%%%%%%%%%%%%%%%%%%

\begin{document}
\begin{center}
Texas Tech University. Applied Mathematics Seminar.

\end{center}

\begin{center}

{\LARGE \uppercase{\textbf{A New Method for Evaluating the Productivity Index of Non-Linear Flows}}}

Akif Ibragimov, Texas Tech University

September 10, 2008

Room: MA 112, Time: 4:00pm

\end{center}


%\begin{abstract}
ABSTRACT.
This work addresses the effects of non-linearity of flows on the value of the productivity index (PI) of the well. Experimental data show that during the dynamical process of hydrocarbon recovery the PI stabilizes to some constant value, which, in general, is a non-linear function of both the pressure drawdown and the production rate.
The case of linear Darcy flows is well understood, and excellent approximate formulae are available for PI in various well-reservoir geometries. To handle the more realistic non-linear situation, the current practice is to solve the nonlinear problem multiple times for different values of production rate, and then to add some ad-hoc corrective parameter(s) in the linear formulae, to reproduce the non-linear nature of the flow.
In the current work we are proposing a rigorous framework to measure PI   of the well for non-linear Forchheimer flows. Our approach, based on recent progress in the modeling of transient Forchheimer flows, uses both analytical and numerical techniques. It provides, for a wide class of reservoir geometries, an accurate relation between the PI for the non-linear Forchheimer and the PI for the linear Darcy flows.
The proposed method of building look-up tables and analytical formulae serves as an effective tool for fast PI evaluation in non-linear cases.

\end{document}
