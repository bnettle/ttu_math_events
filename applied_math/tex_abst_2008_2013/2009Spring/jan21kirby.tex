\documentclass[oneside]{amsart}
\usepackage{amssymb,amsfonts}
\usepackage{amsthm}

\linespread{2}

%%%%%%%%%%%%%%%%%%%%%%%%%%%%%%%%%%%%%%%%%%%%%%%%%%%%%%%%%%%%%%%%%%%%

\begin{document}
\begin{center}
Texas Tech University. Applied Mathematics Seminar.

\end{center}

\begin{center}

{\LARGE \uppercase{\textbf{
From functional analysis to iterative methods (continued)
}}}

Robert Kirby, Texas tech University

January 21, 2009

Room: MA 111, Time: 4:00pm

\end{center}

ABSTRACT. We examine condition numbers, preconditioners, and iterative methods
for finite element discretizations of coercive PDE in the context of
the fundamental solvability result, the Lax-Milgram Lemma.  Working in
this Hilbert space context is justified because finite element
operators are restrictions of Hilbert space operators to
finite-dimensional subspaces.  Moreover, useful insight is gained as
to the relationship between Hilbert space and matrix condition
numbers, and translating Hilbert space fixed point iterations into
matrix computations provides new ways of motivating and explaining
some classic iteration schemes.  In this framework, the ``simplest''
preconditioner for an operator from a Hilbert space into its dual is
the Riesz isomorphism.  Simple analysis gives spectral bounds and
iteration counts bounded independent of the finite element subspaces
chosen.  Moreover, the abstraction allows us not only to consider
Riesz map preconditioning for convection-diffusion-reaction equations
in $H^1$, but also operators on other Hilbert spaces, such as planar
elasticity in $\left(H^1\right)^2$.
\end{document}
