\documentclass[oneside]{article}
\usepackage{amssymb,amsfonts}
\usepackage{amsthm}

\usepackage{hyperref}

\usepackage{fancyhdr}

\usepackage[none]{hyphenat} 
\usepackage[left=.75 in,top=.75 in,right=.75 in,bottom=.75 in,nohead]{geometry}

\pagestyle{fancy}
\lfoot{}
\cfoot{\url{http://www.math.ttu.edu/~dacao/AnalysisSeminar}}
%\cfoot{\textit{http:/$\!$/www.math.ttu.edu/$\!$\raise-1ex\hbox{{\Large\texttt{\char`\~}}}lhoang/AppliedMath/}}
\rfoot{}
\fancyhead{}
\renewcommand{\headrulewidth}{0pt}
\linespread{1.5}

%%%%%%%%%%%%%%%%%%%%%%%%%%%%%%%%%%%%%%%%%%%%%%%%%%%%%%%%%%%%%%%%%%%%

\newcommand{\talktitle}{Asymptotic expansions in Gevrey spaces for solutions of Navier-Stokes equations in periodic domains}

\newcommand{\talkspeaker}{ \textbf{\sc Luan Hoang  }\\ \textit{Texas Tech University}}

\newcommand{\talkdate}{\textbf{Monday, March 05, 2018}}

\newcommand{\timelocation}{\textbf{Room: MATH 108.  Time: 4:00 pm.}}

\newcommand{\talkabstract}{
%
}

\begin{document}

%\vspace*{-2cm}
\begin{center}
{\LARGE
Texas Tech University.  Analysis Seminars.
}
\medskip

\textbf{\Huge {\uppercase{\talktitle}} }

{\LARGE
\talkspeaker\\
\talkdate\\
\timelocation
}
\end{center}

\vspace*{10pt}


%\addtolength{\linewidth}{-1cm}
\begin{center}

\fbox{\parbox{\linewidth}{
\begin{center}
\begin{minipage}[c]{.96\linewidth}
{
\vspace*{.25cm}
{\LARGE \textbf{ABSTRACT.}}
{\Large
We study the long-time behavior of solutions to
the three-dimensional Navier-Stokes equations (NSE) of viscous, incompressible fluids with periodic boundary conditions. The body force is assumed to have a large-time asymptotic expansion in either Sobolev or Gevrey spaces in terms of decaying exponential functions and polynomials.
We will start with proving  basic Gevrey estimates for the nonlinear terms of the NSE.
They are then applied to establish the asymptotic expansions of Foias-Saut-type for all Leray-Hopf weak solutions in the corresponding Sobolev or Gevrey spaces. This
extends the previous results of Foias and Saut in Sobolev spaces for the
case of potential forces. Our proofs, at least in the case of periodic domains, do not require stringent conditions on the time derivatives of the force.
This talk is based on joint research projects
with Vincent Martinez (Tulane University).
}
\vspace*{.25cm}
}
\end{minipage}
\end{center}
}}

\end{center}
\end{document}
