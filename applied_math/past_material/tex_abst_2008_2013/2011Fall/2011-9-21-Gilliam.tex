\documentclass[oneside]{amsart}
\usepackage{amssymb,amsfonts}
\usepackage{amsthm}

\usepackage[none]{hyphenat} 

\setlength{\hoffset}{-.5in}%{-1in}
\addtolength{\hoffset}{1.5cm}
\setlength{\textwidth}{\paperwidth}
\addtolength{\textwidth}{-2\hoffset}
\addtolength{\textwidth}{-2in}

\setlength{\textheight}{\paperheight}
\addtolength{\textheight}{-2in}%to fix pdf missing footer . old val {-.75in}
\setlength{\oddsidemargin}{-0.75cm}
\setlength{\marginparsep}{0cm}
\setlength{\marginparwidth}{0cm}
\setlength{\marginparpush}{0cm}
\addtolength{\textwidth}{-\oddsidemargin}
\addtolength{\textwidth}{-\marginparsep}
\addtolength{\textwidth}{-\marginparwidth}
\addtolength{\textwidth}{-\marginparpush}


\setlength{\voffset}{-.5in}%{-1in}
\addtolength{\voffset}{1cm}
\setlength{\textheight}{\paperheight}
\addtolength{\textheight}{-2\voffset}
\addtolength{\textheight}{-1in}%to fix pdf missing footer . old val {-.75in}


%\textheight=700pt


\linespread{1.75}

%%%%%%%%%%%%%%%%%%%%%%%%%%%%%%%%%%%%%%%%%%%%%%%%%%%%%%%%%%%%%%%%%%%%

\newcommand{\talktitle}{Geometric Output  Regulation for  Nonlinear Distributed Parameter Systems
}

\newcommand{\talkspeaker}{ {\sc David Gilliam}\\
 \textit{Department of Matematics and Statistics, Texas Tech University}}

\newcommand{\talkdate}{Wednesday, September 21, 2011}

\newcommand{\timelocation}{Room: MATH 010.  Time: 4:00pm.}

\newcommand{\talkabstract}{
We consider the  output regulation problem for a special class of nonlinear distributed parameter systems. The geometric approach is  based on the center manifold theorem and as such gives a proof of the local solvability of a wide class tracking and disturbance rejection problems provided one can solve the so-called regulator equations. The regulator equations are a set of equations describing an error zero invariant manifold for the dynamics of the composite closed loop system consisting of the plant (to be controlled) and the exosystem. It is assumed that the neutrally stable  exosystem generated both the signals to be tracked as well as the disturbances that need to be rejected.

The main goal of this talk  is to describe a numerical procedure for the solution of the regulator equations resulting in a control law that solves the desired regulation problem.  We will give a few examples as time permits.
}

\begin{document}

\thispagestyle{empty}

\begin{center}
{\Large
Texas Tech University.  Applied Mathematics Seminar.
}
\medskip

\textbf{\Huge {\uppercase{\talktitle}} }

{\Large
\talkspeaker\\
\talkdate\\
\timelocation
}
\end{center}

\vspace*{10pt}


%\addtolength{\linewidth}{-1cm}
\begin{center}

\fbox{\parbox{\linewidth}{
\begin{center}
\begin{minipage}[c]{.96\linewidth}
{
\vspace*{.25cm}
{\Large \textbf{ABSTRACT.}}
{\huge 
%\talkabstract 
We consider the  output regulation problem for a special class of nonlinear distributed parameter systems. The geometric approach is  based on the center manifold theorem and as such gives a proof of the local solvability of a wide class tracking and disturbance rejection problems provided one can solve the so-called regulator equations. The regulator equations are a set of equations describing an error zero invariant manifold for the dynamics of the composite closed loop system consisting of the plant (to be controlled) and the exosystem. It is assumed that the neutrally stable  exosystem generated both the signals to be tracked as well as the disturbances that need to be rejected.
}

{\huge
The main goal of this talk  is to describe a numerical procedure for the solution of the regulator equations resulting in a control law that solves the desired regulation problem.  We will give a few examples as time permits.
}
\vspace*{.25cm}
}
\end{minipage}
\end{center}
}}


\end{center}
\end{document}
