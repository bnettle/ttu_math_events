\documentclass[oneside]{amsart}
\usepackage{amssymb,amsfonts}
\usepackage{amsthm}

\linespread{2}

%%%%%%%%%%%%%%%%%%%%%%%%%%%%%%%%%%%%%%%%%%%%%%%%%%%%%%%%%%%%%%%%%%%%

\begin{document}
\begin{center}
Texas Tech University.  Applied Mathematics Seminar.

\end{center}

\begin{center}

{\LARGE \uppercase{\textbf{
On regularity at minus infinity of the solution of the parabolic equation of second order
}}}

Akif I. Ibragimov, Texas Tech University

Wednesday, February 11, 2009

Room: MA 111, Time: 4:00pm

\end{center}

ABSTRACT.
In the current presentation I will consider solutions of the first type of the  boundary value problem for parabolic equation in non-cylindrical domain. Following Petrovsky classical approach I will introduce the notion of regularity at minus infinity, and provide criteria in terms of parabolic capacity for the regularity. The results are obtained by using the modified analogue of the Landis lemma of growth in the imbedded cylindrical layers, which are expanding in the space and time.
I will also show how to obtain parabolic analogous of the Phragmén-Lindelöf principle for solutions of the parabolic equation of the second order using lemma of growth.

\end{document}
