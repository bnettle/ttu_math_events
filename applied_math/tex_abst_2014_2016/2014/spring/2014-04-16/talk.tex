\documentclass[12pt]{article}
\usepackage[T2A]{fontenc}
\usepackage[cp1251]{inputenc}
\usepackage[english]{babel}
\usepackage{amsfonts}
\usepackage{amsmath}
\usepackage{amsthm}
\usepackage{latexsym}
\usepackage{amssymb}
%\usepackage{hyperref}

%\newcommand{\esssup}{\mathop{\mathrm{ess\,sup}}}
%\newcommand{\essinf}{\mathop{\mathrm{ess\,inf}}}
%\newcommand{\essosc}{\mathop{\mathrm{ess\,osc}}}
%\renewcommand{\div}{\mathop{\mathrm{div}}}
\newcommand{\dy}{\,\mathrm{d}y}
\newcommand{\dx}{\,\mathrm{d}x}

\newtheorem{theorem}{Theorem}
\newtheorem*{theorem1}{Theorem}
\newcommand{\eps}{\varepsilon}

%\renewcommand{\baselinestretch}{1.5}
\textwidth=158mm
\textheight=232mm
\voffset=-24mm
\pagestyle{empty}
\begin{document}
\begin{center}
{\large\bf On density of smooth functions in weighted Sobolev
spaces with variable exponent.}
\end{center}
\centerline{\textsc{Mikhail D. Surnachev}}
\centerline{\it\small
Computational Aeroacoustics Laboratory,}
\centerline{\it\small
 Keldysh Institute of Applied Mathematics RAS}
\centerline{\small Moscow 125047, Russia}
\centerline{ \texttt{peitsche@yandex.ru}}
\bigskip

In this work we are concerned with the question of density of
smooth functions in weighted Sobolev-Orlicz spaces. In a bounded
Lipshitz domain $\Omega\subset\mathbb{R}^n$ for a weight
(nonnegative measurable function) $\rho\in L^1(\Omega)$ we define
the weighted Sobolev-Orlicz space
$$
W:=\left\{u\in W_0^{1,1}(\Omega)\,:\, \int_\Omega |\nabla
u|^{p(x)}\rho\,\mathrm{d}x<\infty \right\},
$$
with a measurable exponent $p=p(x)>1$, equipped with the norm
$$
\|u\|_W= \inf \left\{\lambda>0: \int_\Omega \left(\frac{|\nabla
u|}{\lambda}\right)^p\rho\,\mathrm{d}x\leq 1\right\}.
$$
We assume that the weight additionally satifsfies $\rho^{-1/p}\in
L^{p'}(\Omega;\,\dx)$, which guarantees completeness of $W$. Next,
we can define another space $H$ as the closure of $C_0^{\infty}$
in $W$.

%The classical question in the theory of Sobolev spaces is whether
%$H=W$. For $p=const$ the following facts are widely known. If
%$\Omega=\mathbb{R}^n$ and $\rho\equiv 1$ the question is simple
%and solved by mollifications. For the standard $W^{1,2}(\Omega)$
%the question was solved in the famous paper of Meyers and Serrin
%entitled $H=W$. They did not require any smoothness of the
%boundary of $\Omega$. If one makes some additional assumptions on
%the smoothness of $\partial\Omega$, it is possible to prove that
%functions smooth up to the boundary are also dense in the Sobolev
%space.

%In the presence of a weight, even in the case of constant $p$, the
%question naturally becomes more complicated. If one tries to
%follows the classical route, the uniform boundedness of mollifying
%operators is required. This question is closely related to the
%boundedness of the Hardy-Littlewood maximal function in the
%corresponding weighted Lebesgue space. For instance, this is
%guaranteed by $\rho$ belonging to the Muckenhoupt class $A_p$. It
%should be noted that the Muckenhoupt condition is generally not
%easy to verify.

%Another approach to approximating a Sobolev function with smooth
%functions  is the so called \textit{Lipschitz truncation}
%technique, based on cutting the original function at some level of
%the maximal function of its gradient and using the McShane
%extension theorem. This technique was successfully applied by a
%number of authors for the study of very weak solutions of elliptic
%PDEs, in mathematical hydrodynamics and so on. In this work we use
%this technique to establish the following sufficient condition for
%density of smooth functions in weighted Sobolev spaces.

Recently V.V. Zhikov proved the following interesting theorem.
\begin{theorem1}[V.V. Zhikov]
Let $\rho=\omega\omega_0$, $p\equiv 2$ and $w_0\in A_2$. Assume
that
$$
\liminf_{t\to\infty} \frac{\left(\int_\Omega \omega^t
\omega_0\dx\right)^{1/t}\cdot \left(\int_\Omega \omega^{-t}
\omega_0\dx\right)^{1/t}}{t^2}<\infty.
$$
Then $H=W$.
\end{theorem1}
First, we obtain a generalization of this result for all constant
$p>1$.
\begin{theorem}\label{teor1}
Let $\rho=\omega\omega_0$, $p=const>1$ and $w_0\in A_p$. Assume
that
$$
\liminf_{t\to\infty} \frac{\left(\int_\Omega \omega^t
\omega_0\dx\right)^{1/t}\cdot \left(\int_\Omega \omega^{-t}
\omega_0\dx\right)^{1/t}}{t^p}<\infty.
$$
Then $H=W$.
\end{theorem}
Next, we extend this result to the case of Sobolev spaces with
variable exponent. We assume that the exponent
$p=p(x):\Omega\rightarrow (1,\infty)$ satisfies
\begin{gather}
1<\alpha<p(x)<\beta<\infty,\label{pc-1}\\
|p(x)-p(y)|\leq \frac{k_0}{\ln \frac{1}{|x-y|}},\quad |x-y|<1.\label{pc-2}
\end{gather}
This is the widely known logarithmic condition introduced
originally by V.V. Zhikov and X.L. Fan in the first half of 1990s.
To state the analogue of Theorem~\ref{teor1} we need to introduce
a generalization of the Muckenhoupt classes for variable
exponents.

\noindent\textbf{Definition.} We say that a nonnegative locally
integrable function (weight) $\omega\in
A_{p(\cdot)}^{loc}(\Omega)$, if
\begin{equation}\label{MuckCondVar}
\sup_{Q} \sup_{x\in Q} \left(\int_Q \omega
\dy\right)^{1/p(x)}\frac{\|\omega^{-1/p}\|_{L^{p'}(Q)}}{|Q|}<\infty,
\end{equation}
where the supremum is taken over all cubes $Q\subset\Omega$ with
faces parallel to the coordinate hyperplanes.

A very important feature of this classes is that they inherit the
self-improvement property of the classical Muckenhoupt classes.
\begin{theorem}\label{self-improv}
Let the exponent $p(\cdot)$ satisfy the logarithmic condition
\eqref{pc-1}-\eqref{pc-2} and $\omega\in
A_{p(\cdot)}^{loc}(\Omega)$. Then $\omega\in
A_{p(\cdot)-\eps}^{loc}(\Omega)$ for some $\eps>0$.
\end{theorem}
The next theorem is the main result of this work.
\begin{theorem}\label{teor3}
Let $\rho=\omega\omega_0$, where $\omega_0\in
A_{p(\cdot)}^{loc}(\Omega)$. Let the exponent $p=p(x)$ satisfy the
logarithmic condition \eqref{pc-1}-\eqref{pc-2}. Let
$$
\liminf_{t\to\infty} \left(\int_{\Omega}\omega^{-t}\omega_0\dx
\right)^\frac{t-1}{t(t+1)} \left(\int_{\Omega}\left(t^{-p(x)}
\omega\right)^{t}\omega_0\dx\right)^{1/t}<\infty,
$$
Then $H=W$.
\end{theorem}

%\noindent\textbf{Remark.} The full theory of Muckenhoupt classes
%with variable exponent was built in \cite{DH13}.

\noindent\textbf{Acknowledgements.} This work was supported by
RFBR, research project 14-01-31341.

\small

\begin{thebibliography}{5}

\bibitem{Zh2011PMA} \textsc{V.V. Zhikov}, \textit{On variational problems and nonlinear elliptic equations with nonstandard growth conditions},
 Journal of Mathematical Sciences \textbf{173}:5 (2011), 463--570.Translated from Problems in Mathematical Analysis \textbf{54}, February 2011, pp. 23-�112.

\bibitem{Zh98MSb} \textsc{V.V. Zhikov}, \textit{Weighted Sobolev spaces}, Sbornik: Mathematics \textbf{189}:8 (1998), 1139�-1170.

\bibitem{Zh04POMI} \textsc{V.V. Zhikov}, \textit{Density of smooth functions in Sobolev-Orlicz spaces}, Journal of Mathematical Sciences \textbf{132}:3 (2006), 285--294.
Translated from Zapiski Nauchnykh Seminarov POMI, Vol. \textbf{310}, 2004, pp. 67�-81.

\bibitem{DH13}  \textsc{L. Diening and P. H\"ast\"o}, \textit{Muckenhoupt weights in variable exponent spaces}, University of Helsinki Preprint (2011).
\verb+http://www.helsinki.fi/~hasto/pp/p75_submit.pdf+
%\bibitem{Samko99} \textsc{S.G. Samko}, Density $C_0^\infty (\mathbb{R}^n)$ in the generalized Sobolev spaces $W^{m,p(x)}(\mathbb{R}^n)$, Dokl. Akad. Nauk \textbf{369}:4
%(1999), 451--454.

%\bibitem{DH13}  \textsc{L. Diening and Peter H\"ast\"o},  \textit{Muckenhoupt weights in variable exponent spaces}, \url{http://www.helsinki.fi/~hasto/pp/p75_submit.pdf}

\end{thebibliography}

\end{document}
