\documentclass[oneside]{article}
\usepackage{amssymb,amsfonts}
\usepackage{amsthm}

\usepackage{hyperref}

\usepackage{fancyhdr}

\usepackage[none]{hyphenat} 
\usepackage[left=.75 in,top=.75 in,right=.75 in,bottom=.75 in,nohead]{geometry}

\pagestyle{fancy}
\lfoot{}
\cfoot{\url{http://www.math.ttu.edu/~dacao/AnalysisSeminar}}
%\cfoot{\textit{http:/$\!$/www.math.ttu.edu/$\!$\raise-1ex\hbox{{\Large\texttt{\char`\~}}}lhoang/AppliedMath/}}
\rfoot{}
\fancyhead{}
\renewcommand{\headrulewidth}{0pt}
\linespread{1.5}

%%%%%%%%%%%%%%%%%%%%%%%%%%%%%%%%%%%%%%%%%%%%%%%%%%%%%%%%%%%%%%%%%%%%

\newcommand{\talktitle}{Mixed boundary value problems for non-divergence type elliptic equations in unbounded domains}

\newcommand{\talkspeaker}{ \textbf{\sc Akif Ibragimov  }\\ \textit{Texas Tech University}}

\newcommand{\talkdate}{\textbf{Monday, March 19, 2018}}

\newcommand{\timelocation}{\textbf{Room: MATH 108.  Time: 4:00 pm.}}

\newcommand{\talkabstract}{
%
}

\begin{document}

%\vspace*{-2cm}
\begin{center}
{\LARGE
Texas Tech University.  Analysis Seminars.
}
\medskip

\textbf{\Huge {\uppercase{\talktitle}} }

{\LARGE
\talkspeaker\\
\talkdate\\
\timelocation
}
\end{center}

\vspace*{10pt}


%\addtolength{\linewidth}{-1cm}
\begin{center}

\fbox{\parbox{\linewidth}{
\begin{center}
\begin{minipage}[c]{.96\linewidth}
{
\vspace*{.25cm}
{\LARGE \textbf{ABSTRACT.}}
{\Large
We investigate the qualitative properties of solutions to the Zaremba type problem in unbounded  domains for non-divergence elliptic equations with possible 
degeneracy  at infinity. 
The main result is a Phragm\'en-Lindel\"of type principle on growth/decay of a solution at infinity depending on both the structure of the Neumann portion of the boundary
and the ``thickness'' of its Dirichlet portion.  The result is formulated in terms of the so-called $s$-capacity 
of the Dirichlet portion of the boundary, while the Neumann boundary should satisfy certain ``admissibility'' condition in a sequence of layers converging to 
infinity.  
This is a joint work with Dat Cao (Texas Tech University) and Alexander I. Nazarov (St. Petersburg Department of Steklov Institute).
}
\vspace*{.25cm}
}
\end{minipage}
\end{center}
}}

\end{center}
\end{document}
