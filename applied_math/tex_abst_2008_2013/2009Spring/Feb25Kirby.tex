\documentclass[oneside]{amsart}
\usepackage{amssymb,amsfonts}
\usepackage{amsthm}

\linespread{1.5}

%%%%%%%%%%%%%%%%%%%%%%%%%%%%%%%%%%%%%%%%%%%%%%%%%%%%%%%%%%%%%%%%%%%%

\begin{document}
\begin{center}
Texas Tech University.  Applied Mathematics Seminar.

\end{center}

\begin{center}

{\LARGE \uppercase{\textbf{
Finite element matrices and Bernstein polynomials
}}}

Robert Kirby, Texas Tech University

Wednesday, February 25, 2009

Room: MA 111, Time: 4:00pm

\end{center}

ABSTRACT.
High-degree finite element methods are typically only made efficient  
in terms of computational cost and memory usage by finding special  
structure that allows operators to be applied with fast algorithms.   
In rectangular element domains, one can frequently build basis  
functions by tensor products of one-dimensional polynomials.  This can  
lead to efficient implementations.

On simplicial element domains, however, far fewer fast algorithms are  
known.  Special bases may be carefully constructed, but these often  
lack certain desireable properties (such as rotational symmetry).   
Working toward the goal of fast, general-purpose simplicial bases with  
good symmetry, I will work the Bernstein polynomials.

For polynomial degree $n$, I will show that finite element mass and  
stiffness matrices in 1d may be applied in $O(n \log(n))$ operations with  
$O(n)$ local store if Bernstein bases are used.  Moreover, I will  
suggest how on the simplex in dimension $d\ge 2$, it is possible to apply  
these operators in $O( n^{1+1/d} \log(n) )$, which up to the log factor  
is the same complexity as rectangular domains with tensor products.    
The trick is Hankel matrices, which up to row ordering admit a  
circulant embedding and hence fast application via the FFT.

This work is preliminary and still theoretical.  I believe there are  
very good opportunities for student projects here, both in terms of  
implementation and extending the results to other kinds of finite  
element bases and operators.

\end{document}
